\documentclass{article}

\usepackage{listings}

\title{Práctica 1 de Desarrollo de Software para Juegos}
\author{Adolfo Hristo David Roque Gámez}

\begin{document}
\maketitle
\section{Desarrollo}
\subsection{Manejo de los límites de Escenario}
El manejo de los límites se realiza en dos ocasiones: una para el jugador y otra para todos los enemigos.

Para el jugador, el fragmento de código mostrado en \ref{lst:ex1} se encarga de mantenerlo en la pantalla. Las coordenadas del jugador están ubicadas en la esquina izquierda del jugador, por eso verifica si es menor o igual a 0 o si es mayor a 736.

A pesar de que el ancho de la pantalla es de 800, el jugador verifica si es mayor de 736 porque está tomando en cuenta el ancho del jugador (64).

\begin{lstlisting}[caption={Código que se encarga de controlar el límite del jugador},label=lst:ex1]
playerX += playerX_change
if playerX <= 0:
    playerX = 0
elif playerX >= 736:
    playerX = 736
\end{lstlisting}
\end{document}